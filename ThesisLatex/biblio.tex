%% You only need to keep one of biblio.bib and biblio.tex, depending which 
%% citation managament style you want to use.

\begin{thebibliography}{10}

\bibitem{ecore}
\newblock Az EMF Ecore meta-metamodell sematikus ábrája.
\newblock \emph{Eclipse Foundation. Xtext Documentation}
\newblock \\ https://eclipse.org/Xtext/documentation/308\_emf\_integration.html
\newblock \\ 2016.04.27

\bibitem{omguml}
\newblock \emph{Object Management Group. OMG Unified Modeling Language Superstructure}
\newblock \\ www.omg.org/spec/UML/
\newblock \\ 2015.11.30.

\bibitem{refactorerl1}
\newblock \emph{Melinda Tóth \& István Bozó:
\newblock Static Analysis of Complex Software Systems Implemented in Erlang},
\newblock \\ Central European Functional Programming Summer School Fourth Summer 
School, CEFP 2011, Revisited Selected Lectures, Lecture Notes in Computer
Science (LNCS), Vol. 7241, pages 451-514, Springer-Verlag, ISSN: 0302-9743,
\newblock 2012.

\bibitem{refactorerl2}
\newblock \emph{István Bozó, Dániel Horpácsi, Zoltán Horváth, Róbert Kitlei, Judit Kőszegi, Máté Tejfel, Melinda Tóth:
\newblock RefactorErl - Source Code Analysis and Refactoring in Erlang},
\newblock \\ Proceedings of the 12th Symposium on Programming Languages and Software
Tools, ISBN 978-9949-23-178-2, pages 138-148,
\newblock \\ Tallin, Estonia,
\newblock \\ 2011.10.

\bibitem{erlangdocs}
\emph{Erlang Programming Language},
\newblock \\ http://www.erlang.org/
\newblock \\ 2015.11.30

\end{thebibliography}
