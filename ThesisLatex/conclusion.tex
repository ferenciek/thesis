The main goal of this thesis, as stated at the beginning of this work, is to analyze a quality of programs written in the Erlang programing language by using RefactorErl tool. To accomplish that goal, it became necessary to provide a general overview of Erlang, RefactorErl static analysis tool and developed metrics. In this thesis the software quality, software metrics and some of tools for measure software quality metrics has been studied and analyzed.

Erlang is a functional language designed for highly parallel, scalable applications requiring high uptime. RefactorErl is a static source code analysis and transformation tool for Erlang providing several software metrics. The tool is developed by the Department of Programming Languages and Compilers at the Faculty of Informatics, Eötvös Loránd University, Budapest, Hungary. Among the features of RefactorErl is included a metric query language which can support Erlang developers in everyday tasks such as program comprehension, debugging, finding relationships among program parts, etc.

In this thesis we presented developed framework which analyze git repositories with Erlang code files. The new component is built on RefactorErl static analysis tool and actively uses its feature of calculating different metrics of Erlang modules and functions. This frawework allows to draw plots which show change of metrics with software evolving from version to version. The plots can be saved for future usage as a pictures in PNG format. 

The main focus of the component is to help Erlang developers with analyzing their projects using plots. Also, it was important to test the developed component on some projects and after that to analyze the measurments. For this purpose have been choosen three different projects from git. The experimental results allows to find changes (increasing the line of code number, char of code number, using otp library and etc.) in code. Visualisation helps with finding patterns and improving of the code quality.

It is safe to say that the main goals of this thesis were successfully achieved. The usage of software metrics is within an organization and it usage is expected to have a beneficial effect on software organizations by make software quality more visible.