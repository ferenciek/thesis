Requirements for the quality of the product being developed have rapidly increased in recent years. From the beginning of software product development, developers have been striving to monitor quality. Software metrics and their visualization are two important features of measurement systems. 

In this thesis, we introduce framework for analysing of quality of programs written in the Erlang programing language which built on the top of the RefactorErl static analyzis tool. Our goal is to analyse
projects and then prepare the results of the measurements.

Erlang is a functional language designed for highly parallel, scalable applications requiring high uptime. Several industrial and open source products were implemented in Erlang, therefor tools to measure the complexity, qulity of the source code are highly desirable. 

RefactorErl is a static source code analysis and transformation tool for Erlang providing several software metrics. The tool is developed by the Department of Programming Languages and Compilers at the Faculty of Informatics, Eötvös Loránd University, Budapest, Hungary. 
Among the features of RefactorErl is included a metric query language which can support Erlang developers in everyday tasks such as program comprehension, debugging, finding relationships among program parts, etc.
Software metrics provide a means to extract useful and measurable information about the structure of a software system. The results of these evaluation methods can be used to indicate which parts of a software system need to be reengineered.

This thesis is organized as follows: Chapter 2 is needed for understanding the main concepts of software metrics. Chapter 3 covers the necessary background on Erlang programming language, RefactorErl tool, defined metrics  in this tool and description developed framework. Chapter 4 illustrates measurements and findings of some projects. Chapter 5 describes some related works. Chapter 6 consists conclusion about complited work.

