%%%%%%%%%%%%%%%%%%%%%%%%%%%%%%%%%%%%%%
%% Header section of Latex document %%
%%%%%%%%%%%%%%%%%%%%%%%%%%%%%%%%%%%%%%

\documentclass[runningheads,a4paper]{report}
%% @author Daniel Lukacs, dlukacs@caesar.elte.hu, 2017

\usepackage[a4paper]{geometry}
\usepackage{t1enc}
\usepackage[utf8]{inputenc}
\usepackage{lmodern}

\usepackage[title,titletoc]{appendix}
\usepackage[section]{placeins}
\usepackage{relsize}

\usepackage[normalem]{ulem} %% Provides underlining.
\usepackage{caption} %% Provides captions.
\usepackage{mdframed} %% Provides frames around text and equations.
\usepackage{tikz-cd} %% Provides diagram drawing environment.
\usepackage{adjustbox} %% Provides additional tools to resize content.
% \usepackage[magyar]{babel} %% Provides foreign language support.

%%%%
%% Provides math related environments and directives.
\usepackage{amssymb}
\usepackage{amsthm}
\usepackage{amsmath}
\usepackage{latexsym}

%% See http://tex.stackexchange.com/questions/43835/conflict-between-amsthm-and-some-other-package
\let\proof\relax 
\let\endproof\relax

%%%%
%% Provides table environments and related directives.
\usepackage{array}
\usepackage{tabulary}
\usepackage{tabularx}
\usepackage{multirow}
\usepackage{hhline}

%%%%
%% Provides figure environments and related directives.
\usepackage{graphicx}
\makeatletter
\def\maxwidth#1{\ifdim\Gin@nat@width>#1 #1\else\Gin@nat@width\fi}
\def\maxheight#1{\ifdim\Gin@nat@height>#1 #1\else\Gin@nat@height\fi}
\makeatother

\usepackage{fancyvrb}
\usepackage{rotating}

%%%%
%% Provides environment to display source code.
\usepackage{listings} 
\lstset{ 
    literate=%
        {á}{{\'a}}1
        {é}{{\'e}}1
        {í}{{\'i}}1
        {ó}{{\'o}}1
        {ö}{{\"o}}1
        {ő}{{\H{o}}}1
        {ú}{{\'u}}1
        {ü}{{\"u}}1
        {ű}{{\H{u}}}1
        {Á}{{\'A}}1
        {É}{{\'E}}1
        {Í}{{\'I}}1
        {Ó}{{\'O}}1
        {Ö}{{\"O}}1
        {Ő}{{\H{O}}}1
        {Ú}{{\'U}}1
        {Ü}{{\"U}}1
        {Ű}{{\H{U}}}1
    } %% Customization of listings env., to enable non-English accents.

\lstset{
   frame=single,
   basicstyle=\small,
   language=Erlang,
   numbers=left,
   firstnumber=1,
   numberfirstline=true,
%  basicstyle=\ttfamily,
%  columns=fullflexible,
%   keepspaces=true,
} %% Customization of listings environment


%%%%
%% Provides environment to display pseudocode.
\usepackage{algorithm}% http://ctan.org/pkg/algorithms
\usepackage{algpseudocode}% http://ctan.org/pkg/algorithmicx

\newcommand{\repeatcaption}[2]{%
  \addtocounter{figure}{-1}%
  \renewcommand{\thefigure}{\ref{#1}}%
  \captionsetup{list=no, labelformat=simple, labelsep=colon}%
  \captionof{figure}{#2}%
} %% Customization: Using the same figure twice with no new number. See http://tex.stackexchange.com/a/200229

%%%%
%% Provides directives to display followable URL references.
\usepackage{url}
\usepackage{hyperref}
\hypersetup{
  hidelinks,
  linkbordercolor = {0 0 1},
}

%% Customization: Followable links to appendix references.
\makeatletter
\appto{\appendices}{\def\Hy@chapapp{Appendix}}
\makeatother


%%%%
%% Custom document formatting.

% \renewcommand{\abstract}{ \begin{center}\textbf{Abstract}\end{center}}

\setcounter{tocdepth}{2}

\setlength{\parskip}{\baselineskip}%
\setlength{\parindent}{0pt}%

\makeatletter
\renewcommand\subsubsection{\@startsection{subsubsection}{3}{\z@}%
                       {-18\p@ \@plus -4\p@ \@minus -4\p@}%
                       {4\p@ \@plus 2\p@ \@minus 2\p@}%
                       {\normalfont\normalsize\bfseries\boldmath
                        \rightskip=\z@ \@plus 8em\pretolerance=10000 }}
\makeatother


%%%%
%% Custom theorem environments.
\newtheorem{mydef}{Definition}
\newtheorem{myexamp}{Example}

%%%%
%% Custom symbol definitions and abbreviations.
\makeatletter
\providecommand{\leadsfrom}{%
  \mathrel{\mathpalette\reflect@squig\relax}%
}
\newcommand{\reflect@squig}[2]{%
  \reflectbox{$\m@th#1\leadsto$}%
}
\makeatother

\renewcommand{\labelitemi}{$\circ$}
\newcommand{\edge}[1]{\stackrel{\bf{#1}}{\rightarrow}}
\newcommand{\ledge}[1]{\stackrel{\bf{#1}}{\leftarrow}}
\newcommand{\rel}[1]{\stackrel{\bf{#1}}{\leadsto}}
\newcommand{\trel}[1]{\stackrel{\bf{#1}}{\leadsto^*}}
\newcommand{\lrel}[1]{\stackrel{\bf{#1}}{\leadsfrom}}

\newcommand{\eqname}[1]{\tag*{#1}}% Tag equation with name

\newcommand{\nv}[0]{node(v)}
\newcommand{\ruleref}[1]{(\S\ref{#1})}
\newcommand{\apxref}[1]{(Appendix \ref{#1}.)}
\newcommand{\apxrefm}[3]{(Appendix \ref{#1}., \ref{#2}. és \ref{#3}.)}



%%%%%%%%%%%%%%%%%%%%%%%%%%%%%%%%%%%%
%% Body section of Latex document %%
%%%%%%%%%%%%%%%%%%%%%%%%%%%%%%%%%%%%

\begin{document}
% \title{The title of your thesis}
% \thispagestyle{empty}
% \begin{center}
% {\Huge TDK dolgozat}\\[0.5cm]
% {\bf Név} \\[1cm]
% \end{center}


\begin{titlepage}
  \noindent
  \begin{minipage}{0.25 \textwidth}
    \includegraphics[height=40mm]{figures/cimer.png}
  \end{minipage}
  \hfill
  \begin{minipage}{0.67 \textwidth}
    \large
    Eötvös Loránd University \\
    Faculty of Informatics \\
    Department of Programming Languages and Compilers \\
    
  \end{minipage}

  \vfill

  \begin{center}
    {\LARGE \bfseries Analysing the changes of software metric values with RefactorErl}
    %% \\[1.5cm]
    %% {\Large TDK dolgozat}
    %% \\[3cm]
    \\[6cm]
    \begin{minipage}[t]{0.45 \textwidth}
      \emph{Supervisor:} \\[0.25 \baselineskip]
      {\large Melinda Tóth} \\[0.5 \baselineskip]
      Assistant lecturer
    \end{minipage}
    \begin{minipage}[t]{0.45 \textwidth}
      \begin{flushright}
        \emph{Author:} \\[0.25 \baselineskip]
        {\large Marina Konoreva} \\[0.5 \baselineskip]
        Computer Science MSc \\ %% The name of your program
        2. year
      \end{flushright}
    \end{minipage}
  \end{center}

  \vfill

  \begin{center}
    \large Budapest, 2019
  \end{center}
\end{titlepage}

%%%%%%%%%%%%%%%%%%%%%%%%%%%%%%%%%
%% Content sections start here %%

\begin{abstract}
%% While you can write all your content here in the main file, it's recommended
%%   to keep your content into separate files. The \input directive simply
%%   copies here the text of the pointed line.
Nowadays, the requirements for development speed and software quality are significantly increasing. The use of a flexible architecture and various design techniques certainly can improve the quality of development, but formal quality criteria, such as code metrics, showing the quantitative characteristics of a software system in various dimensions, still remain relevant. In evaluating the duration and complexity of software development, various metrics are used.
Software metrics and their visualization are two important features of   measurement systems. 

The  goal  of  my  thesis  work  is to  analyse  open  source  software  with  RefactorErl,log  the changes of metric value, visualize it and conclude the findings.


\end{abstract}

\tableofcontents

\chapter{Introduction}

Nowadays, the requirements for development speed and software quality are significantly increasing. The use of a flexible architecture and various design techniques certainly can improve the quality of development, but formal quality criteria, such as code metrics, showing the quantitative characteristics of a software system in various dimensions, still remain relevant. In evaluating the duration and complexity of software development, various metrics are used.
Software metrics and their visualization are two important features of   measurement systems. 

In this thesis, we introduce a module for analysing of quality of programs written in the Erlang programing language. Our goal is to analyse
projects and then prepare the results of the analysis, so it can be added to the RefactorErl static analysis framework for Erlang.

\chapter{Software metrics}
This chapter presents the introduction to the software metrics as the important tool for evaluation of quality and productivity of the software development product. Also the classification of the software metrics, description the different measurements of software metrics in general and the most popular software metrics tools are included in this chapter.


\section{Definition of Software Metrics}

Software metrics are the attributes of the software systems that deals with the measurements of the software product and process by which it is developed ~\cite{metrix}.
 
A software metric is a measure of characteristics of software which are countable or quantifiable. The importance of software metrics is valuable for many reasons, including planning work items, measuring software performance, measuring productivity, and many other uses.

Software developers must recognize the principles of software metrics that involve cost,schedule find quality goals, quantitative goals, comparison of plans with actual performance throughout development, monitoring data trends for indication of likely problems, metrics presentation, and investigation of data values.

There are some metrics within the process of software development, that are all related to each other. Metrics can be related to the four functions of management:

\begin{itemize}
	\item Controlling.
	\item Planning.
	\item Improving.
	\item Organising.
\end{itemize}

The aim of analyzing and tracking software metrics is to find out the quality of the particular product or process, enhance its quality and forecast the quality once the software development project is done. On a more detailed level, software development managers try to:

\begin{itemize}
	\item Manage workloads.
	\item Increase return on investment (ROI).
	\item Reduce overtime.
	\item Identify areas of improvement.
	\item Reduce costs.
\end{itemize}

These goals can be polished by providing information and clarity overall the organization about complex software development projects. Metrics are an essential value of quality assurance, performance, management, debugging, and estimating costs, and they are valuable for both development team leaders and developers:

\begin{itemize}
	\item Teams of software developers can use software metrics to interact between the status of
	software development projects, pinpoint and address issues, and observe, improve on,
	and manage their workflow better.
	\item Managers of software can use software metrics to track, identify, communicate and prioritize any
	issues to foster better team productivity. This permits effective management and allows
	prioritization and assessment of problems within software development projects. The
	sooner managers can find software problems, the easier and less-expensive the process of troubleshooting.
\end{itemize}

Software metrics provide an assessment of the impact of decisions made through the process of development of software projects. This helps managers prioritize and assess performance goals and objectives.

\section{Classification of software metrics}

Software metrics are broadly classified as product metrics and process metrics as shown in Figure \ref{fig:classification} ~\cite{metrics2} \footnote{Nenad Medvidovic André van der Hoek, Ebru Dincel. Using service utilization metrics to assess and improve product line architectures. Proceedings of the 9th International Symposium on Software Metrics, 2003.}.
\begin{figure}[ht]
	\centering
	\includegraphics[width=\textwidth]{figures/classification.png}
	\caption{Classification of software metrics.}
	\label{fig:classification}
\end{figure}

Process metrics are numerical values that depict a software process such as the amount of time require to debug a module~\cite{metrics2}. They are measures of the software development process, such as: overall development time and type of methodology used. Process metrics are collected across all projects and over long periods of time. Their intent is to provide indicators that lead to longterm software process improvement.

Product metrics can be used to analyze and check the software project. These type of metrics calculates the complexity of the software design size of the final program number of pages of documentation produced. They allow a software project manager to do the following:

\begin{itemize}
	\item[--] decrease the development time by making the necessary modifications to avoid delays and potential problems and risks.
	\item[--] project quality assessment and change the technical approach to get better quality.
\end{itemize}

Product metrics are measures of the software product of any stage of its development,from requirements to installed system ~\cite{metrics2}. Product metrics may measure: the complexity of the software design, the size of the final program, the number of pages of documentation produced.
Product metrics can be internal or external. External attributes of an entity can be measured only with respect to how the entity relates with the environment and therefore can be measured only indirectly. For example, reliability, an external attribute of a program, does not depend only on the program itself but also on the compiler, machine and user. Productivity, an external attribute of a person, clearly depends on many factors such as the kind of process and the quality of the software delivered. Internal product metrics can be measured only based on the entity and therefore the measures are direct. For example, size is an internal attribute of any software document.

Internal product metrics are subdivided in two categories: cognitive
complexity metrics and structural complexity metrics. Cognitive complexity metrics measure the effort required by developers to understand a system. They aim at discovering the cause of the complexity, which requires understanding human mental processes and details of the software system under development. Structural complexity metrics use the interactions within and among modules to measure a system’s complexity. One of the oldest and most commonly used structural complexity metrics is the number of source lines of code.

Another classification of software metrics is as follows:

\begin{enumerate}
	\item Objective metrics.
	\item Subjective metrics
\end{enumerate}

Objective metrics always results in identical values for a given metric as measured by two or more qualified observers. Whereas subjective metrics are those that even qualified observers may measure different values for a given metric since their subjective judgment is involved in arriving at the measured value.
%%____________________________________________________________________________

\section{Types Of Software Metrics}

It is now apparent that software metrics are important in software engineering. Symons stated that "a reliable and credible method for measuring the software development cycle is needed that has a reasonable theoretica1 basis and that produces results that practitioners can trust" ~\cite{symons}. Hence, software metrics were used to measure a wide range of software developing activities.

\subsection{Size-Oriented metrics}

Size-oriented metrics are used to analyze the quality of software.

\textbf{Lines of Code (LOC)}

Lines Of Code (also possible SLOC - Source Lines of Code) is a metric generally used to measure a software program or codebase according to its size. It represents how many lines of source
code exist in the application, class, methord or namespace. LoC can be used for: checking the size of code units and estimating the size of project. LOC is the simplest one but very popular.
There are two major types of SLOC measures: 

\begin{description}
	\item[Physical SLOC (LOC)] Physical SLOC is a number of lines in the source code of the program including comment.
	\item[Logical SLOC (LLOC)] Logical LOC tries to calculate the number of "statements", but their specific definitions are tied to specific computer languages.
\end{description}

Physical SLOC measures are sensitive to logically irrelevant formatting and style conventions, while logical LOC is less sensitive to formatting and style conventions. Unfortunately, SLOC measures are often stated without giving their definition, and logical LOC can often be significantly different from physical SLOC.

\subsection{Object-oriented metrics}

Chidamber and Kemerer have specified several metrics for object-oriented designs ~\cite{ck}. All of these metrics are referred not to the whole system, but the separate class.

\textbf{Number Of Methods (NOM)}

The Number Of Methods metric is used to measure the average count of all class operations per class. A class must have some, but not an excessive number of operations. This information is useful when identifying a lack of primitiveness in class operations (inhibiting re-use), and in classes which are little more than data types.

\textbf{Number Of Children (NOC)}

NOC metric measures the number of subclasses which belong to a class. This metric calculates the class hierarchy.

NOC metric is closely relevant with DIT metric, which is more better because it supports for reusing methods through inheritance. The first metric calculates the number of child classes, DIT measures the depth of the class.
Inheritance levels can be used to gain the depth and reduce the breadth.

A high number of NOC means the following: 
\begin{itemize}
	\item Wrong abstraction of the parent class. 
	\item The reuse of the base class is high. The form of reuse is inheritance.
	\item Wrong using of subclasses. In this case, it is needed to introduce a new level of inheritance with newly grouped related classes.
	\item Base class needed to be more tested.
\end{itemize}

Classes high hierarchy have more subclasses then classes with low hierarchy. The number of children gives an idea of the potential influence a class has on the design ~\cite{ck}.

\textbf{Weighted Methods per Class (WMC)}

This metric calculates a sum of complexities all defined methods in a class. WMC shows the complexity of whole class. This measure helps to indicate the development and maintenance effort for the class. Classes with a large number of WMC can often be refactored into several classes.

A class with a high value of WMC and a high number of NOC indicates complexity at the top of the class hierarchy. A sign of poor design is the potential impact of the base class on a large number of subclasses.

\textbf{Coupling Between Object classes (CBO)}

CBO classes metric demonstrates the number of classes coupled to a given class. This coupling can happen through:
\begin{itemize}
	\item Properties or parameters. 
	\item Method call. 
	\item Method arguments or return types.
	\item Class extends.
	\item Variables in methods
\end{itemize}

Coupling among classes is crucial for a system to do useful work, but redundant coupling makes the system more complicated to maintain and reuse. At the project or package level, this metric displays the average number of classes used per class. A measure of coupling is useful to determine how complex the testing of various parts of a design are likely to be ~\cite{ck}.

\textbf{Depth of Inheritance Tree (DIT)}

Depth of Inheritance Tree (DIT) measures the maximum length of a path from the current class to the root class in the inheritance structure of a system. DIT calculates how many super-classes can affect a class. DIT is applicable only to object-oriented systems.

If a class is on the deep level in the hierarchy, the more methods and variables it tends to inherit, what makes it more complex. Deep trees indicate big complexity of the design. Inheritance is a key for complexity managing, really, not for its increasing. As a positive factor, deep trees promote reuse because of method inheritance. Deeper trees constitute greater design complexity, since more methods and classes are involved ~\cite{ck}.

C\&K suggested the following consequences based on the depth of inheritance:
\begin{itemize}
	\item Deeper trees establish large design complexity, since more classes and methods are involved
	\item If a class is on the deep level in the hierarchy, the more methods and variables it tends to inherit, what makes it more complex to foresee its behavior
	\item If a particular class is on the deep level in the hierarchy, there is a great chance of the possible reuse of inherited methods 
\end{itemize}

\textbf{Response For a Class (RFC)} 

The Response for Class (RFC) metric measures the total number of methods that can probably be executed as a response to a message received by some object of a class. This number calculates as the sum of the methods of the class, and all distinct methods are called directly within the class methods. Additionally, it counts inherited methods, but not overridden methods, because only one method of a particular signature will always be accessible for an object of a given class.

A large RFC is an indicator of more faults. Classes that have a high RFC are more complex and more difficult to understand. Testing and debugging for these classes is also complicated. A worst-case value for possible responses will assist in the appropriate allocation of testing time. If a large number of methods can be invoked on response to a message, the testing and debugging of the class becomes more complicated since it requires a greater level of understanding required on the pan of the tester ~\cite{ck}.
%%___________________________________________________
\subsection{Complexity metrics}

Complexity is an important aspect for software quality assessment and must be appropriately addressed in service-oriented architecture ~\cite{complexity}. One of the key aims of complexity  metrics is to predict modules that are fault-prone post-release ~\cite{complexity2}. These metrics are one of the most difficult software metrics for understanding.

\textbf{McCabe's Cyclomatic Complexity (MVG)}

To determine the complexity of a software, McCabe suggests a "mathematical technique that will provide a quantitative basis for modularisation and allow us to identifY software modules that will be difficult to test or maintain" ~\cite{shepperd}. McCabe's cyclomatic complexity ~\cite{mc} is a software quality metric that shows the complexity of a software program. Complexity is inferred by summarizing the number of linearly independent paths through the program. The higher the number the more complex the code.

A pragmatic approximation to this can be found by counting language keywords and operators which introduce extra decision outcomes.
%%_______________________________
\subsection{Structural Metrics}

\textbf{Fan-In and Fan-Out metrics (FIN and FOUT)}

It is a structural metrics which measures inter-module complexities. 
\begin{description}
	\item[Fan-out] Is the number of modules that are called by a given module.
	\item[Fan-in] Is the number of modules that call a given module.
\end{description}

Fan-out and fan-in metrics reflect structure dependency ~\cite{fanin}.
These structural metrics were first defined by Henry.
These metrics can be applied both at the module level and function level. These metrics just put a number on how complex is interlinking of different modules or functions. Unlike Cyclomatic complexity, you cannot put a number and say it cannot go beyond this number. This is used just to size up how difficult it will be to replace a function or module in your application and how changes to a function or module can impact other functions or modules. Sometimes you can put the restriction on the number of Fan-Out.

%%______________________________________________________

\subsection{Cohesion metrics}
Cohesion is an important software quality attribute and high cohesion is one of the characteristics of well-structured software design ~\cite{cohesion}.
Cohesion metrics analyze the connection between the methods of a class.
Module cohesion indicates relatedness in the functionality of a software module ~\cite{cohesion2}.

\textbf{Lack of Cohesion in Methods (LCOM)}

LCOM calculates the number of cohesiveness present, how well a system was designed and how complex a class is. LCOM is a count of the number of method pairs whose similarity is zero, minus the count of method pairs whose similarity is not zero. LCOM is probably the most controversial and argued over of the C\&K metrics.

C\&K's rationale for the LCOM method was as follows:
\begin{itemize}
	\item Lack of cohesion implies classes should probably be split into two or more subclasses.
	\item The cohesiveness of methods within a class is desirable since it promotes encapsulation.
	\item Low cohesion increases complexity, thereby increasing the likelihood of errors during the development process.
	\item Any measure of disparateness of methods helps identify flaws in the design of classes. 
\end{itemize}

Although there is a fair amount of debate about how to calculate LCOM and it features in a lot of metrics sets an increasing number of researchers to suggest that it is not a particularly useful metric. Perhaps this is also reflected in there being a fair amount of debate about how to calculate LCOM but very little on how to interpret it and how it fits in with other metrics. 

\textbf{Tight and Loose Class Cohesion (TCC and LCC)}

TCC (Tight Class Cohesion) and LCC (Loose Class Cohesion) metrics measure the relative number of directly-connected pairs of methods and the relative number of directly- or indirectly- connected pairs of methods.
The Tight Class Cohesion metric measures the cohesion between the public methods of a class. That is the relative number of directly connected public methods in the class. Classes having a low cohesion indicate errors in the design.

TCC considers two methods to be connected if they share the use of at least one attribute. A method uses an attribute if the attribute appears in the method’s body or the method invokes directly or indirectly another method that has the attribute in its body. The higher TCC and LCC, the more cohesive and thus better the class.

In this Section, we focused our attention on all possible software metrics. In the next Chapter, we will define metrics which are used to describe various aspects of Erlang projects. 
%%______________________________________________________________________

\section{Software Metric Tools}
There are a lot of software metrics have been developed and numerous tools exist to gather the metrics from program representations. This large number of tools allows a user to choose the tool best suited for user requirements, for example, its handling, tool support, cost etc. This is accepted that the metrics computed by the metric tools are the same for all the metric tools. One can think of a software metric tool as a program which implements a set of software metrics definitions. It allows accessing a software system according to the metrics by extracting the required entities from the software and providing the corresponding metric values. There are
some criteria for selecting the proper metric tools as the availability of the software tools can make confusion. One such criterion is that the tools must have to calculate any form of software metrics. Majority metric tools are available for Java programs. Many tools are just code counting
tool, they basically count the variants of the lines of code (LOC) metric. The specific criteria areas follow language: Java (source or bytecode), metrics: well-known object-oriented metrics on class level, license: freely available.

\subsection{CCCC}

CCCC is a small command-line tool which generates metrics from source code of C or C++ project ~\cite{cccc}. The tool outputs a simple HTML website with information about all your sources. It generates reports on different metrics, for example, lines of code (LOC) and metrics suggested by Chidamber\&Kemererand Henry\&Kafura. CCCC is distributed as a freeware and is released in the form of source code. Users have to compile the program on their own and to modify the source code to reflect their interests and preferences.

CCCC can process every file which was provided via the command line. It is possible to use standard wildcard process. For every file, CCCC will check the extension of the filename, and if the extension is recognized as one from supported languages, the particular parser will parse this file. As every file is parsed, certain constructs recognition will cause records to be inserted into an internal database. After all files have been processed, the software will generate the report in HTML format. This report is depended on on the contents of the internal database. By default settings the main HTML report is produced to the file cccc.htm in a subfolder called .cccc of the current working folder. It includes detailed reports on every module (for example, C++ or Java class) detected by the analysis run.

As an addition to HTML reports and summary, the run of the program will generate the corresponding summary and reports in XML format. Also, the file called cccc.db will be created. This file will represent a dump of the internal database of the software in a special format with '@' symbol as a delimiter. It is chosen because this symbol is one of the few which can not appear in C/C++ source code.

The report consists of a number of tables which identify the modules in the submitted files and covering:
\begin{enumerate}
	\item Measures the number and the relationships type.
	\item Measures the procedural volume and complexity and functions of every module. 
	\item A summary report over the whole codebase processed of the measures described above.
	\item Identification of any parts of the source code submitted which the program can not parse.
\end{enumerate}

This tool can measure the following metrics:
\begin{itemize}
	\item Fan-In Fan-Out (FIN and FOUT).
	\item Lines of Code (LOC). 
	\item Number Of Children (NOC).
	\item Weighted Methods per Class (WMC).
	\item McCabe's Cyclomatic Complexity (MVG).
	\item Number Of Methods (NOM).
\end{itemize}

\subsection{Chidamber\&Kemerer}

The program counts Chidamber and Kemerer object-oriented metrics by introspection the bytecode of compiled Java files ~\cite{ck1}. It is an open source command line tool. The program counts the following six metrics for each class, and displays them on its standard output with the name of the class.

This tool can measure the following metrics:

\begin{itemize}
	\item Depth of Inheritance Tree (DIT).
	\item Weighted Methods per Class (WMC).
	\item Numbe r Of Children (NOC).
	\item Coupling Between Object classes (CBO).
	\item Lack of Cohesion in Methods (LCOM).
	\item Response For a Class (RFC).
\end{itemize}

\subsection{Analyst4j}
Analyst4j is built on the Eclipse platform and can be downloaded as a standalone Rich Client Application or also as an Eclipse
IDE plugin ~\cite{analyst4j}. Its features are search, metrics analyzing, quality analyzing, report generating for Java programming.
Analyst4j software is most popular to find out the quality-related metrics. This tool is based on Chidamber\&Kemerer metrics.

This tool can measure the following metrics:
\begin{itemize}
	\item Weighted Methods per Class (WMC).
	\item Lines of Code (LOC). 
	\item Coupling Between Object classes (CBO).
	\item Depth of Inheritance Tree (DIT).
	\item Response For a Class (RFC).
	\item Number Of Children (NOC).
	\item Lack of Cohesion in Methods (LCOM).
	\item Number Of Methods (NOM).
\end{itemize}

\subsection{OOMeter}

OOMeter is a software metric tool for measuring the quality attributes of Java and C\# source code and UML models, stored in XMI format ~\cite{meter}. OOMeter has a rich collection of object-oriented software metrics. This is the Eclipse plugin. It provides a querying language for object-oriented code similar to SQL which allows to search for measure code metrics, bugs etc.

OOMeter provides an interface for users to define custom metrics through Java classes that implement a certain interface. It supports export of metric results to a number of formats, including XML, HTML, delimited text, Microsoft Excel, etc ~\cite{meter2}.

This tool can measure the following metrics:

\begin{itemize}
	\item Weighted Methods per Class (WMC).
	\item Lines of Code (LOC). 
	\item Coupling Between Object classes (CBO).
	\item Depth of Inheritance Tree (DIT).
	\item Response For a Class (RFC).
	\item Numbe r Of Children (NOC).
	\item Tight Class Cohesion (TCC).
	\item Lack of Cohesion in Methods (LCOM).
\end{itemize}



\subsection{Eclipse Metrics plugin 1.3.6}

This is an open source dependency analyzer and metrics calculation plugin for Eclipse IDE ~\cite{ecl}. The plugin is also provided integrated as an EasyEclipse package. The plugin computes the various metrics and displays it in the integrated view.


This tool can measure the following metrics:

\begin{itemize}
	\item Weighted Methods per Class (WMC).
	\item Lines of Code (LOC). 
	\item Numbe r Of Children (NOC).
	\item Depth of Inheritance Tree (DIT).
    \item Number Of Methods (NOM).
\end{itemize}

\subsection{Eclipse Metrics plugin 3.4}
The eclipse plugin 3.4 developed by Lance Walton is also integrated with Eclipse and is available for all Java projects developed using the IDE ~\cite{ecl1}. It is an open source tool. It counts various metrics in the moment of build cycles and shows warnings via the problem view of metrics range violations.

This tool can measure the following metrics:
\begin{itemize}
	\item Weighted Methods per Class (WMC).
	\item Lines of Code (LOC). 
	\item Lack of Cohesion in Methods (LCOM).
	\item Depth of Inheritance Tree (DIT).
\end{itemize}

\subsection{Semmle}

Semmle is the platform for analyzing that produces a detailed report of the code base for one or more software projects ~\cite{semmle}. For every project that it analyzes, it calculates artifacts against rules that check for good practice. Analysis can be scheduled to run on a regular basis. The copy of the source code is checked out from the repository for analysis as part of this process. The code, and related artifacts is checked against rules, defined using queries, to identify any alerts. Finally, metrics are calculated and data can be imported from third-party systems used by your company. A database is created, containing detailed information about the artifacts and every alert.

This tool can measure the following metrics:
\begin{itemize}
	\item Lack of Cohesion in Methods (LCOM).
	\item Depth of Inheritance Tree (DIT).
	\item Number Of Methods (NOM).
	\item Numbe r Of Children (NOC).
	\item Response For a Class (RFC).
\end{itemize}

\section{Measuring functional languages}

In the previous section has been described software metrics for object-oriented languages. However, software metrics developed for imperative and object-oriented languages can also be used for measuring in functional programming languages like Erlang and Haskell. 

Some  of  the  measurement  techniques  from  imperative  and object-oriented  languages  may  transfer  quite  cleanly  to  functional  languages,  for  instance  the  path  count  metric  which  counts  the  number  of  execution  paths through  a  piece  of  program  code,  but  some  of  the  more  advanced features  of  functional  programming  languages  may contribute  to  the  complexity  of  a  program  in  ways  that  are  not  considered  by  traditional  imperative  or  object  oriented 
metrics ~\cite{fp}. 

We can use the same metrics because several constructs as a class, a module, and a library are similar. All of this structures can be consider like collections of functions. If the chosen metric does not take the distinctive properties of these constructs into account (variables, method overrides, dynamic binding, visibility etc.), then it can be applied to these apparently diverse constructs ~\cite{metrics3}.

The dissimilarity between functional and imperative languages are in the difference in the level of nesting of blocks and control structures, in several ways of connecting certain functions (for example, data flow and call graph), inheritance instead of cohesion and simple cardinality metrics(lines of code, char of code).

Another difference functional programming languages from imperative languages is there are some constructs and properties that can be used only in functional programming languages as list comprehensions, pattern matching, referential transparency of pure functions, currying, laziness of expression evaluation.

While these features raise the expressive power of functional languages,
most of the existing complexity metrics require some changes before they become applicable to functional languages ~\cite{metrics3}.

There are general metrics are acceptable for functional languages:
\begin{itemize}
	\item \textbf{Branches of recursion}. This metric allows measuring how many times did the function call itself
	\item \textbf{Fun expressions and message passing constructs}.
	\item \textbf{Return points of a function}.
	\item There is possible to calculate metrics on a single clause of a function.
	\item There is possible to calculate metrics on a single clause of a function.
	\item \textbf{Otp used}. This metric allows measure OTP behaviors.
\end{itemize}

In the next chapter will be described all developed metrics for Erlang in details.

\chapter{Sofrware metrics in erlang}
In this chapter, we present a developed framework which allows to analyze Erlang programs and visualize calculated metrics. The first two sections give a brief summary of Erlang, RefactorErl static analysis tool and developed metrics.

\section{An introductory glimpse at the Erlang programming language}
Erlang is a functional programming language which provides runtime designed for highly parallel, scalable software requiring high uptime. It is designed for development of robust systems that can be distributed between many different computers in a network. Erlang is kinda of similar to Java in the case that it uses a virtual machine and also supports multithreading. 

\textbf{Variables}
Erlang provides dynamic data types, allowing programmers to develop system
components (such as message dispatchers) that do not care what type of data they are handling and others that strongly enforce data type restrictions or that decide how to act based on the type of data they receive. Variables must start with a capital letter or an underscore, and are composed of letters, digits,	 and underscores.

\textbf{Data types}
Erlang has:
\begin{enumerate}
	\item Integers of unlimited size.
	\item Floats.
	\item Strings, placed within double quotes: "It is some string.".
	\item Atoms. An atom is an element by itself. It starts with a lowercase letter and is built of letters, digits, and underscores, or it is any string placed within single quotes: atom1, 'Atom 2'.
	\item Lists are a comma-separated sequence of some values placed within brackets: [abc, 123, "It is some string"]. 
	\item Tuples are a comma-separated sequence of some values placed within braces: {abc, 123, "It is some string"}.
	\item Records are not a separate data type but are just tuples with keys associated with each value. They are declared in a file and defined (given specific values) in the program.
	\item Binaries are placed within double angle brackets: <<0, 128, 128, 255>>, <<"It is some string">>, <<X:7, Y:5, Z:1>>. Binaries are series of bits; the number of bits in a binary has to be a multiple of 8.
	\item References are globally unique values.
	\item Pids stand for process identifiers which are the "names" of processes.
\end{enumerate}

Figure \ref{fig:example_erlang}features the source of a small Erlang program
called example that demonstrated recursive list manipulation.
\begin{figure}[h]
	\begin{lstlisting}[extendedchars=true, language=Erlang, basicstyle=\footnotesize\ttfamily, keywordstyle=\color{red}]
	-module(example). 
	-export([max/1, min/1, sum/1]).
	
	%% Find the maximum of a list.
	max([H|T]) -> max2(T, H).
	max2([], Max) -> Max;
	max2([H|T], Max) when H > Max -> max2(T, H);
	max2([_|T], Max) -> max2(T, Max).
	

	%% Find the minimum of a list.
	min([H|T]) -> min2(T,H).
	min2([], Min) -> Min;
	min2([H|T], Min) when H < Min -> min2(T,H);
	min2([_|T], Min) -> min2(T, Min).
	
	%% Find the sum of all the elements of a list.
	sum(L) -> sum(L,0).
	sum([], Sum) -> Sum;
	sum([H|T], Sum) -> sum(T, H+Sum).
	\end{lstlisting}
\caption{A simple module in Erlang.}
\label{fig:example_erlang}
\end{figure}

Erlang source files consist of a section containing meta-information about the module represented by the file(all functions in Erlang must be defined in
modules.), and a list of functions that are either exposed to the users of this module (with the -export attribute), or are only defined for internal use inside the module.  

A subtle element of all three functions is that every function needs to have an initial value to start counting with. In the case of sum/2, we use 0, as we’re doing addition, and given \texttt{X = X + 0}, the value is neutral, so we can’t mess up the calculation by starting there. If we were doing multiplication, we would use 1 given \texttt{X = X * 1}. 

The functions min/1 and max/1 can’t have a default starting value. If the list were only negative numbers and we started at 0, the answer would be wrong. So we need to use the first element of the list as a starting point.

\section{The RefactorErl static analysis tool} 

RefactorErl~\cite{refactorerl1, refactorerl2} is an open-source static source code analyzer and transformer tool for Erlang, developed by the Department of Programming Languages and Compilers at the Faculty of Informatics, Eötvös Loránd University, Budapest, Hungary. The phrase "refactoring" means a preserving source code transformation, so while you change the program structure you do not alter its behavior. RefactorErl was built to refactor Erlang programs.

The main focus of RefactorErl is to support daily code comprehension tasks of Erlang developers~\cite{refactorerl}. It can analyze the structure of the refactored program - based on the syntactic rules of the underlying programming language - and it can also collect and use semantical information about the source code.

\subsection{Metrics in the RefactorErl}

A metric query language is incorporated into RefactorErl~\cite{refactorerl}. Metric queries can be executed from the console interface or can be used as properties in semantic query language which is available from every interface.

Table \ref{tab:metrics_ref} shows all the implemented metrics in RefactorErl tool. There are two columns in this table: the first column gives the information about the name of the metric, and the second column shows the for which node the metric is available.

\begin{table}[!htb]
	\centering
	\caption{Implemented metrics in RefactorErl}
	\label{tab:metrics_ref}
	\noindent\adjustbox{max width=0.9\textwidth}{
		\begin{tabular}{|c|c|}
			\hline
			\textbf{Name of the metric}  & \textbf{Node type} 
			\\
			\hline
			module sum					& module
			\\
			
			\hline
			line of code				& module/function
			\\
			
			\hline
			char of code				& module/function
			\\
			
			\hline
			number of fun				& module
			\\
			
			\hline
			number of macros			& module
			\\
			
			\hline
			number of records			& module
			\\
			
		  	\hline
			included files				& module
			\\	
				
		  	\hline
		  	imported modules			& module
		  	\\	
		  	
		  	\hline	
		  	number of funpath			& module
		  	\\	

		  	\hline
		  	function calls in			& module
		  	\\
		  	
		  	\hline
		  	function calls out			& module
		  	\\	
		  	
		  	\hline
		  	cohesion					& module
		  	\\	  	
		  	
		  	\hline
		  	function sum				& function
		  	\\	  	
		  			  	
		  	\hline
		  	max depth of calling		& module/function		  			  
		  	\\
		  	
		  	\hline
		  	max depth of cases			& module/function		  			  
		  	\\		
		  	  	
		  	\hline
		  	min depth of cases			& module/function		  			  
		  	\\		  	

		  	\hline
		  	max depth of structs		& module/function		  			  
		  	\\		  			

		  	\hline
		  	number of funclauses		& module/function		  			  
		  	\\			
	
		  	\hline
		  	branches of recursion		& module/function		  			  
		  	\\			

		  	\hline
		  	calls for function			& function		  			  
		  	\\			
			
		  	\hline
		  	calls from function			& function		  			  
		  	\\	

		  	\hline
		  	number of funexpr			& module/function		  			  
		  	\\
		  	
		  	\hline
		  	number of messpass			& module/function		  			  
		  	\\		  	

		  	\hline
		  	fun return points			& module/function		  			  
		  	\\	
		  	
		  	\hline
		  	average size				& module/function		  			  
		  	\\		  		  	

		  	\hline
		  	max length of line			& module/function		  			  
		  	\\
		  	
		  	\hline
		  	no space after comma		& module/function		  			  
		  	\\	
		  	
		  	\hline
		  	is tail recursive			& function		  			  
		  	\\		  	

		  	\hline
		  	mcCabe						& module/function		  			  
		  	\\	
		  			  	
		  	\hline
		  	otp used						& module		  			  
		  	\\			  	
		  	\hline		  		  			  				
		\end{tabular}}
	\end{table}

\textbf{module\_sum}

The sum of the chosen complexity structure metrics measured on the functions of the module. The proper metrics adjusted in a list can be implemented in the desired number and order~\cite{refactorerlm}.

\textbf{line\_of\_code}

The number of lines of part of the text, function, or module. The number of empty lines is not included in the sum. As the number of lines can be measured on more functions, or modules and the system is capable of returning the sum of these, the number of lines of the whole loaded program text can be enquired~\cite{refactorerlm}.

\textbf{char\_of\_code}

The number of characters in a program script. This metric is capable of measuring both the codes of functions and modules and with the help of aggregating functions we can enquire the total and the average number of characters in a cluster, or in the whole source text~\cite{refactorerlm}.

\textbf{number\_of\_fun}

This metric gives the number of functions implemented in the concrete module, but it does not contain the number of non-defined functions in the module~\cite{refactorerlm}.

\textbf{number\_of\_macros}

This metric gives the number of defined macros in the concrete module or modules. It is also possible to inquire the number of implemented macros in a module~\cite{refactorerlm}.

\textbf{number\_of\_records}

 This metric gives the number of defined records in a module. It is also possible to inquire the number of implemented records in a module~\cite{refactorerlm}.

\textbf{included\_files}

This metric gives the number of visible header files in a module~\cite{refactorerlm}.

\textbf{imported\_modules}

This metric gives the number of imported modules used in a concrete module. The metric does not contain the number of qualified calls (calls that have the following form: module:function)~\cite{refactorerlm}.

\textbf{number\_of\_funpath}

The total number of function paths in a module. The metric, besides the number of internal function links, also contains the number of external paths or the number of paths that lead outward from the module. It is very similar to the metric called cohesion~\cite{refactorerlm}.

\textbf{function\_calls\_in}

Gives the number of function calls into a module from other modules. It can not be implemented to measure a concrete function. For that, we use the calls\_for/1 function~\cite{refactorerlm}.

\textbf{function\_calls\_out}

Gives the number of every function call from a module towards other modules. It can not be implemented to measure a concrete function. For that, we use the calls\_from/1 function~\cite{refactorerlm}.

\textbf{cohesion}

The number of call-paths of functions that call each other. By call-path we mean that an f1 function calls f2 (e.g. f1()->f2().). If f2 also calls f1, then the two calls still count as one call-path~\cite{refactorerlm}.

\textbf{function\_sum}

The sum calculated from the functions complexity metrics characterizes the complexity of the function. It can be calculated using various metrics together~\cite{refactorerlm}.
	 
\textbf{max\_depth\_of\_calling}

The length of function call-chains, namely the chain with the maximum depth~\cite{refactorerlm}.
	 
\textbf{max\_depth\_of\_cases}

 Gives the maximum of case control structures embedded in case of a concrete function (how deeply are the case control structures embedded). In case of a module, it measures the same regarding all the functions in the module. Measuring does not break in case of case expressions, namely when the case is not embedded~\cite{refactorerlm}.

\textbf{min\_depth\_of\_cases}

 Gives the minimum of the maximums of case control structures embedded in case of a concrete function (how deeply are the case control structures embedded). In case of a module it measures the same regarding all the functions in the module. Measuring does not break in case of case expressions, namely when the case is not embedded into a case structure. However, the following embedding does not increase the sum~\cite{refactorerlm}.

\textbf{max\_depth\_of\_structs}

 Gives the maximum of structures embedded in function (how deeply are the block, case, fun, if, receive, try control structures embedded). In case of a module it measures the same regarding all the functions in the module~\cite{refactorerlm}.

\textbf{number\_of\_funclauses}

Gives the number of functions clauses. Counts all distinct branches, but does not add the functions having the same name, but different arity, to the sum~\cite{refactorerlm}.

\textbf{branches\_of\_recursion}

Gives the number of a certain function's branches, how many times a function calls itself, and not the number of clauses it has besides definition~\cite{refactorerlm}.

\textbf{calls\_for\_function}

This metric gives the number of calls for a concrete function. It is not equivalent to the number of other functions calling the function, because all of these other functions can refer to the measured one more than once~\cite{refactorerlm}.

\textbf{calls\_from\_function}

This metric gives the number of calls from a certain function, namely how many times does a function refer to another one (the result includes recursive calls as well)~\cite{refactorerlm}.

\textbf{number\_of\_funexpr}

Gives the number of function expressions in a module. It does not measure the call of function expressions, only their initiation~\cite{refactorerlm}.

\textbf{number\_of\_messpass}

In the case of functions, it measures the number of code snippets implementing messages from a function, while in case of modules it measures the total number of messages in all of the modules functions~\cite{refactorerlm}.

\textbf{fun\_return\_points}

The metric gives the number of the functions possible return points (or the functions of the given module)~\cite{refactorerlm}.

\textbf{average\_size}

The average value of the given complexity metrics (e.g. Average branches\_of\_recursion calculated from the functions of the given module)~\cite{refactorerlm}.

\textbf{max\_length\_of\_line}

It gives the length of the longest line of the given module or function~\cite{refactorerlm}.

\textbf{average\_length\_of\_line}

It gives the average length of the lines within the given module or function~\cite{refactorerlm}.

\textbf{no\_space\_afte\r\_comma}

It gives the number of cases when there are not any whitespaces after a comma or a semicolon in the given module's or function's text~\cite{refactorerlm}.

\textbf{is\_tail\_recursive}

It returns with 1 if the given function is tail recursive; with 0, if it is recursive, but not tail recursive; and -1 if it is not a recursive function (direct and indirect recursions are also examined). If we use this metric from the semantic query language, the result is converted to tail\_rec, non\_tail\_rec or non\_rec atom~\cite{refactorerlm}.

\textbf{mcCabe}

McCabe cyclomatic complexity metric. We define it based on the control flow graph of the functions with the number of different execution paths of a function, namely the number of different outputs of the function~\cite{refactorerlm}.

\textbf{otp\_used}

Gives the number of OTP callback modules used in modules~\cite{refactorerlm}.

\section{Metric visualisation module for WEB2 interface of RefactorErl}

The section describes the main concept and details of the developed framework for RefactorErl static analysis tool. The first part introduces the main features of the software. The next part describes which tools were used in the developing process of the module. In the last part, we can read how to use the software step-by-step.

\subsection{Description of the software}

The program which was developed allows analyzing git repositories with Erlang code files. The component is built on RefactorErl static analysis tool and actively uses its feature of calculating different metrics of Erlang modules and functions. It has convenient user-friendly web interface with repository structure as a folder tree and a canvas where plots can be observed. These plots contain information how particular metric was changing with repository evolution. The plot can be saved for future use as a picture in PNG format.

The main features of the software are:
\begin{itemize}
	\item Drawing plots which show the change of metrics with software evolving from version to version.
	\item Analyzing modules and functions separately.
	\item Choosing separate files for the analyzing.
	\item User-friendly web interface.
\end{itemize}

The main focus of the project is to help Erlang developers with analyzing their projects using plots. Visualization helps with finding patterns and improving the code quality. 

\subsection{Used tools}

The interface of the program is the AngularJS component which uses NVD3 library for plot rendering. Metrics data is stored in DETS tables at the stage of calculation in Erlang code. This data passes as JSON objects when Erlang function communicates with javascript code. For saving plots as pictures the saveSvgAsPng library is used.

\textbf{NVD3}

This library currently under development of a team of software engineers at Novus Partners company. NVD3 is the D3 based JavaScript library. It allows creating beautiful and reusable charts in web applications.

It has wonderful features for data visualization with nice-looking charts such as the usual box-plot, line and bar charts and fancier candlestick and sunburst charts. If you need a lot of functionality in a JavaScript chart plotting library, NVD3 is the very nice option for your project.

\textbf{AngularJs}

The new component was created for the existing system using this javascript web framework.

AngularJS (also written as Angular.js) is an open-source JavaScript-based front-end web application framework mainly developed by Google and by a community of individuals and corporations to solve many of the challenges which can be observed in the development of single-page applications.

The AngularJS framework starts his work by reading the Hypertext Markup Language (HTML) page, which has extra tag attributes embedded into it. Angular transforms these attributes as directives to bind input or output parts of the page to a model that is defined by standard JavaScript variables. The values of these JavaScript variables can be manually set in the code, or fetched from static or dynamic JSON resources. 

\textbf{DETS tables}

Data of calculated metrics for modules and functions stored in two different tables: mods\_metrics and funs\_metrics.

DETS is Disk Erlang Term Storage. DETS tables store tuples, with access to the elements given through a  key field in the tuple. The tables are implemented using hash tables and binary trees, with different representations providing different kinds of collections~\cite{erland_o'reilly}.

\textbf{JSON}

Data, which stored in Dets tables, transforms into JSON objects when Erlang code interact with JavaScript code. JSON (JavaScript Object Notation) is a simple format of data-interchange. It is a text format that is absolutely language independent but it uses a convention that is familiar to programmers of the C-family of languages which includes C, C++, C\#, Java, JavaScript, Python, Perl, and others. This property makes JSON an ideal language of data-interchange.

JSON is built on two data structures:
\begin{itemize}
	\item A collection of name/value pairs. In various languages, this is realized as an object, struct, record, hash table, dictionary, associative array or keyed list.
	\item An ordered list of values. Usually, this is realized as an array, list, vector, or sequence.
\end{itemize}

\textbf{saveSvgAsPng}

This small library is used for saving SVG plots as PNG pictures. Despite its small size, it has a lot of different options such a choosing particular background, font, scale etc. 

\subsection{The typical workflow}

For using the component WEB2 interface should be run. It can be done with this command executed in RefactorErl shell:

\begin{lstlisting}[frame=none, numbers=none]
	ri:start_web2([{yaws_path, PATH-TO-YAWS}]).
\end{lstlisting}

This command will run WEB2 interface available on localhost:8001 by default. After loginning the component can be accessed in the metrics tab as shown in Figure \ref{fig:metrics_interface}

\begin{figure}[h]
	\includegraphics[height=80mm]{figures/metrics.png}
	\caption{Component interface.}
	\label{fig:metrics_interface}
\end{figure}

The path to git repository should be provided in "Git repository path" input. After clicking the "Check repository" button the folder will be analyzed. If it is not valid the alert Figure \ref{fig:metrics_alert} will be shown.  

\begin{figure}[h]
	\includegraphics[height=80mm]{figures/alert.png}
	\caption{Alert showed because the provided folder is not a valid repository.}
	\label{fig:metrics_alert}
\end{figure}

In another case, the folder tree will be available where separate files can be chosen for future analyzing as shown in Figure \ref{fig:metrics_files}. Also, there is a possibility to delete some files chosen by mistake.

\begin{figure}[h]
	\includegraphics[height=80mm]{figures/files.png}
	\caption{Repository tree.}
	\label{fig:metrics_files}
\end{figure}

The final step is pressing the "Analyze" button. It will start calculating metrics for all versions of the repository and progress will be shown on screen Figure \ref{fig:metrics_analyze}. 

\begin{figure}[h]
	\includegraphics[height=80mm]{figures/analyze.png}
	\caption{The process of repository analyzing.}
	\label{fig:metrics_analyze}
\end{figure}

After analysis is done the menu with choosing parameters of drawing the plot will be available as shown at Figure \ref{fig:metrics_plot}. It consists of two selection lists. The first one is the list where we can choose the type of item for plots drawing (module or function). Depends on the chosen type the select boxes with available metrics will differ. 

For modules: module sum, line of code, char of code, number of fun, number of macros, number of records, included files, imported modules, number of funpath, function calls in, function calls out, cohesion, max depth of calling, max depth of cases, min depth of cases, max depth of structs, number of funclauses, branches of recursion, number of funexpr, number of messpass, fun return points, average size, max length of line, no space after comma, mcCabe, otp used.

For functions: line of code, char of code, function sum, max depth of calling, max depth of cases, min depth of cases, max depth of structs, number of funclauses, branches of recursion, calls for function, calls from function, number of funexpr, number of messpass, fun return points, average size, max length of line, no space after comma, is tail recursive, mcCabe. Another list is the list of all items which can be chosen for metrics plot drawing. 

After the plot is shown the "Save" button appears which can be pressed to save SVG plot in a PNG file. 

\begin{figure}[h]
	\includegraphics[height=80mm]{figures/plot.png}
	\caption{The example of plot.}
	\label{fig:metrics_plot}
\end{figure}

\chapter{Measurements and findings}
The aim of this chapter is to test and analyze projects from git by using the developed module for RefactorErl. Git commit logs hold all the change history. This is an excellent source to observe trends and patterns about projects. All projects selected for the experiment are written in Erlang and have more than 40 commits.

\section{Iron}

This project is functional Erlang Toolkit. Iron is released under the MIT license. It can do the foolowing:
\begin{itemize}
	\item Count with coerce equality, count with a custom predicate.
	\item Find with coerce equality, find with a custom predicate.
\end{itemize}

This project has just only one source code file with 68 commits. The link to the repository is \url{https://github.com/elementerl/iron}.

We can see that with the version number increase the line of code number and char of code number also grow on Figure \ref{fig:loc_iron} and in Figure \ref{fig:char_iron}.

\begin{figure}[h]
	\centering
	\includegraphics[height=45mm]{figures/loc_iron.png}
	\caption{Effective Line of code for fe.erl file.}
	\label{fig:loc_iron}
\end{figure}

\begin{figure}[h]
	\centering
	\includegraphics[height=45mm]{figures/char_iron.png}
	\caption{Char of code for fe.erl file.}
	\label{fig:char_iron}
\end{figure}

As shown in Figure \ref{fig:otp_iron} developer started to use otp library after 45th version. 

\begin{figure}[h]
	\centering
	\includegraphics[height=45mm]{figures/otp_iron.png}
	\caption{Otp used for fe.erl file.}
	\label{fig:otp_iron}
\end{figure}

The Figure \ref{fig:mcCabe_iron} shows an overview of the evolution of the overall McCabe cyclomatic complexity. We can observe that the complexity keeps increasing.

There was a spike in 45th version, because some functions were called to an other module, however author removed changes in 46th version. There was also an insignificant drop in complexity in 28th version and a little increase in 29th version. The  complexity decreases  when  the  extracted  piece  of code occurred more than one time and the complexity of the function is more than one ~\cite{mcCabe}. 

In Figure \ref{fig:max_depth_of_cases_iron} we can see that the metric \textbf{max\_depth\_of\_cases} is 1, but before it was 0, therefore developer stopped using case inside another case for this version of the software. We can find the same trend on plot with \textbf{mcCabe} metric where it appears as decreasing of the program complexity.  

\begin{figure}[h]
	\centering
	\includegraphics[height=45mm]{figures/mcCabe_iron.png}
	\caption{McCabe cyclomatic complexity metric for fe.erl file.}
	\label{fig:mcCabe_iron}
\end{figure}

\begin{figure}[h]
	\centering
	\includegraphics[height=45mm]{figures/max_depth_of_cases_iron.png}
	\caption{Max depth of cases for fe.erl file.}
	\label{fig:max_depth_of_cases_iron}
\end{figure}

The average length of the line was not stabilized until 35th version with gradually decreasing from 50 symbols to 38 symbols in Figure \ref{fig:average_length_of_line_iron}.

\begin{figure}[h]
	\centering
	\includegraphics[height=45mm]{figures/average_length_of_line_iron.png}
	\caption{Average length of line for fe.erl file.}
	\label{fig:average_length_of_line_iron}
\end{figure}

As we can see in Figure \ref{fig:number_of_macros_iron} and Figure \ref{fig:number_of_records_iron} there are not defined macroses and records in the whole iron project.

\begin{figure}[h]
	\centering
	\includegraphics[height=45mm]{figures/number_of_macros_iron.png}
	\caption{Number of macros for fe.erl file.}
	\label{fig:number_of_macros_iron}
\end{figure}

\begin{figure}[h]
	\centering
	\includegraphics[height=45mm]{figures/number_of_records_iron.png}
	\caption{Number of records for fe.erl file.}
	\label{fig:number_of_records_iron}
\end{figure}
%%____________________________________________________________________-

\section{Erlang chat }

This project is multi-user chat written in Erlang. It has nine source code files and 45 commits. The link to the repository is \url{https://github.com/bildeyko/erlangChat}

For this project, we analyzed the module \textbf{websocket\_handler.erl}. This module consists of 6 functions.

We can see an increasing number of lines of code in Figure \ref{fig:loc_chat} and an increasing number of characters in a program text in Figure \ref{fig:char_of_code_chat}.

\begin{figure}[h]
	\centering
	\includegraphics[height=45mm]{figures/loc_chat.png}
	\caption{Effective Line of code for module websocket\_handler.erl.}
	\label{fig:loc_chat}
\end{figure}

\begin{figure}[h]
	\centering
	\includegraphics[height=45mm]{figures/char_of_code_chat.png}
	\caption{Characters of the code for module websocket\_handler.erl.}
	\label{fig:char_of_code_chat}
\end{figure}

In Figure \ref{fig:chat} we can see that the author started using message passing from the 9th version of his software.

\begin{figure}[h]
	\centering
	\includegraphics[height=45mm]{figures/chat.png}
	\caption{The number of message passing for module websocket\_handler.erl.}
	\label{fig:chat}
\end{figure}

As shown in Figure \ref{fig:chat5} there was increase in the length of the longest line of code in 9th version. 
\begin{figure}[h]
	\centering
	\includegraphics[height=45mm]{figures/chat5.png}
	\caption{Max length of line for module websocket\_handler.erl.}
	\label{fig:chat5}
\end{figure}

McCabe’s cyclomatic complexity metric measurement guarantees that developers are sensitive to the fact that programs with high McCabe numbers, for example, more than 10 are likely to be hard for understanding and accordingly have a higher probability of defects containing within the code base. The tested module has the cyclomatic complexity number which increased to 30 in the last versions as shown in Figure \ref{fig:mcCabe}.

\begin{figure}[h]
	\centering
	\includegraphics[height=45mm]{figures/mcCabe.png}
	\caption{
	McCabe cyclomatic complexity metric for module websocket\_handler.erl.}
	\label{fig:mcCabe}
\end{figure}

In Figure \ref{fig:chat2} shows that developer used otp library functions but after 9th version reconsidered to use them.

\begin{figure}[h]
	\centering
	\includegraphics[height=45mm]{figures/chat2.png}
	\caption{Otp used for websocket\_handler.erl.}
	\label{fig:chat2}
\end{figure}

The developed framework allows measuring and visualizing metrics for a module and also for each function in the module. For example, in this module developer use message passing from the 9th version as we mentioned above. The Figure \ref{fig:chat3} shows that there was discovered function in which the author actively used message passing.

\begin{figure}[h]
	\centering
	\includegraphics[height=45mm]{figures/chat3.png}
	\caption{Number of message passing for function websocket\_handle/3.}
	\label{fig:chat3}
\end{figure}

Visualizing of metrics helps to find which functions have been changed, added or deleted. As shown in the Figure \ref{fig:chat3} the developer slightly changed the function \textbf{terminate/3} by adding two lines of code in the 9th version.

\begin{figure}[h]
	\centering
	\includegraphics[height=45mm]{figures/chat3.png}
	\caption{
		Number of message passing for function websocket\_handle/3.}
	\label{fig:chat3}
\end{figure}
 
To summarize findings we can assume that most of these changes were done in the 9th version.

%%______________________________________________________

\section{prx}

This project is an Erlang library for Unix process management and system programming tasks. Code from all project is divided into 4 modules. 

The project provides:

\begin{itemize}
	\item Reliable operating system process management by mapping Erlang processes to a hierarchy of system processes.
	\item Beam-friendly interface for system calls and other POSIX operation.
	\item Operations for processes isolation like jails and containers.
	\item An interface for separation operations with privileges for processes restriction.
\end{itemize}


The link to the repository is \url{https://github.com/msantos/prx}. This project has 201 commits. 

For this project, we tested the module \textbf{prx.erl}. 

As in previous two experiments, at the beginning, we started to measure the \textbf{LOC} metric. This metric helps us to see the changes all of the project from over time. This result is shown in Figure \ref{fig:line_of_code_prx}.

\begin{figure}[h]
	\centering
	\includegraphics[height=45mm]{figures/line_of_code_prx.png}
	\caption{Effective Line of code for module prx.erl.}
	\label{fig:line_of_code_prx}
\end{figure}

Another important metric is \textbf{cohesion} metric. Modules with large cohesion number is preferable because high cohesion is associated with many desirable attributes of software such as robustness, reusability, reliability, and understandability. Otherwise, low cohesion is associated with undesirable attributes, for example, being difficult to reuse, maintain, test, or even understand. The Figure \ref{fig:cohesion_prx} and shows the decreasing of the calculated metric. 

\begin{figure}[h]
	\centering
	\includegraphics[height=45mm]{figures/cohesion_prx.png}
	\caption{The cohesion of the module prx.erl.}
	\label{fig:cohesion_prx}
\end{figure}

When developer started to use otp library in this project in 80th version, as we can see in Figure \ref{fig:otp_prx}, the McCabe cyclomatic complexity metric was rapidly increased in Figure \ref{fig:McCabe}. If complexity is increasing dramatically between versions, it is an indication of logic 
being added. 

\begin{figure}[h]
	\centering
	\includegraphics[height=45mm]{figures/mccabe.png}
	\caption{McCabe for the module prx.erl.}
	\label{fig:McCabe}
\end{figure}

When we are comparing results of cohesion and McCabe cyclomatic complexity metrics, we can conclude that low cohesion increases complexity, thereby increasing the likelihood of errors during the development process.

\begin{figure}[h]
	\centering
	\includegraphics[height=45mm]{figures/otp_prx.png}
	\caption{OTP used for the module prx.erl.}
	\label{fig:otp_prx}
\end{figure}

Apart from analyzing the changes of the module we also can observe the changes of functions. The Figure \ref{fig:find/2} shows that the function \textbf{find/2} was used only in one version and later was renamed or deleted.

\begin{figure}[h]
	\centering
	\includegraphics[height=45mm]{figures/find2.png}
	\caption{Effective Line of code for function find/2.}
	\label{fig:find/2}
\end{figure}

One of the features of functional programming languages is the presence of tail recursion. It is a special form of recursion where the last operation of a function is a recursive call ~\cite{tail}.
The metric \textbf{is\_tail\_recursive} returns with 1, if the given function is tail recursive; with 0, if it is recursive, but not tail recursive and -1 if it is not a recursive function. As shown in Figure \ref{fig:tail1} we can see that developer got rid of this function from 37th version until 40th version.

\begin{figure}[h]
	\centering
	\includegraphics[height=45mm]{figures/filter2.png}
	\caption{Is tail recursive metric for function filter/2.}
	\label{fig:tail1}
\end{figure}

Also we can calculate how many times a function calls itself by using \textbf{branches\_of\_recursion} metric for all module. The Figure \ref{fig:br} illustrates that there were created more recursive functions after 140th version.

\begin{figure}[h]
	\centering
	\includegraphics[height=45mm]{figures/br.png}
	\caption{Branches of recursion for module prx.erl.} 
	\label{fig:br}
\end{figure}

In RefactorErl there is a \textbf{function\_sum} metric which can be calculated using these metrics together: \textbf{line\_of\_code}, \textbf{char\_of\_code}, \textbf{number\_of\_funclauses}, \textbf{branches\_of\_recursion},
\textbf{mcCabe}, \textbf{calls\_for\_function}, \textbf{calls\_from\_function}, \textbf{fun\_return\_points}, \textbf{number\_of\_messpass}. Metrics \textbf{calls\_for\_function} and \textbf{calls\_from\_function} available only for functions. We can notice the dependency of \textbf{function\_sum} (Figure \ref{fig:function_sum_task4}) on two metrics changes: number of lines of code (Figure \ref{fig:task4}) and calls from the function (Figure \ref{fig:calls_from_function_task4}).

\begin{figure}[h]
	\centering
	\includegraphics[height=45mm]{figures/function_sum_task4.png}
	\caption{Function sum function task/4.} 
	\label{fig:function_sum_task4}
\end{figure}

\begin{figure}[h]
	\centering
	\includegraphics[height=45mm]{figures/calls_from_function_task4.png}
	\caption{Calls from the function for function task/4.} 
	\label{fig:calls_from_function_task4}
\end{figure}

\begin{figure}[h]
	\centering
	\includegraphics[height=45mm]{figures/task4.png}
	\caption{Effective Line of code for function task/4.} 
	\label{fig:task4}
\end{figure}

\chapter{Related work}
Nowadays, the need for the visualization of software quality metrics has been rapidly increased. Software metrics help developers and companies to check and analyze information about the performance, quality of code and cost of software data. it helps to find out and fix errors in the early stages of development.

In this chapter will be described some tools for visualization of software quality metrics and tool for code analysis.

\section{Open Source tool "METRIX"}

This tool can compute different software quality metrics. METRIX is able to evaluate software written in C and ADA languages and many metrics can be considered for software evaluation (the different metrics will be described in detail in the next section~\cite{metrix}. 

Different diagrams enable the user to visualize numeric data. Graphics like line charts, scatter plots and histograms are common. 
There are two less-common classes of diagrams: 
\begin{itemize}
	\item The radar plots, also called Kiviat diagrams.
	\item The city map diagram, mostly used to represent the cartography of cities.
\end{itemize}
One specific feature of METRIX is to use those two types of visualization for constructing signature and cartography of source codes.
 
\textbf{Kiviat diagrams}
The Kiviat diagram visualizes information through polar coordinates. The distance between the point and the origin is associated to the value user want to represent, while the angle between two points is constant, this constant is uninformative and calculated to uniformly distribute the different points. Moreover, all metric points are linked together, making a plain polygon, which shapes the specificity of the data that the user wants to represent. Of course, this diagram is not adequate to represent only one or two metrics, it requires at least three values to be pertinent. Values may be of the same metrics, representing the metric measured on different parts of the source code. Values may be of very different metrics, computed  with heterogeneous units. That enables the user to merge multivariate data on the same diagram. On Kiviat diagrams, values must be strictly positive. 

In figure\ref{fig:metrix} shows an example of two Kiviat diagrams. These diagrams represent coding different metric values for two functions. 

\begin{figure}[h]
	\centering
	\includegraphics[height=70mm]{figures/metrix.png}
	\caption{An example of Kiviat diagrams.}
	\label{fig:metrix}
\end{figure}

\textbf{City map diagrams}
This diagram is mainly used in urbanism, to represent cities and their buildings in three dimensions. As a software project  can contain a very large number of source files, code functions  and code variables, but this visualization diagram adequately suits for complexity visualization issue. 

The treemap diagram is used in computing to represent in two dimensions source code metrics with rectangles. Some authors used variants of this class of visualization diagrams to represent  values with esthetic considerations. City map diagrams are indeed multi-parameters diagrams: each “building” is associated with an n-tuple of numeric values.  

This tool also has a graphical user interface with a single Window containing three tabs.

The first tab called “Calc” and enables the user to indicate the source file(s) to evaluate and to parameterize the metrics to compute. It avoids using hand-start scripts with a prompt, but the open source aspect of the tool allows advanced users to reuse the measuring scripts manually and/or to integrate them into other software.

The second tab called “Csv” presents the numeric values as a list, with a tab per function and per file. The user may export these values to a spreadsheet application, in order to represent  the values with common visualization graphs, like scatter plots, line plots, etc.

The third tab called “Plots” and provides a way to visualize the city map diagrams for the source code, the signatures of functions and the comparisons between functions through radar/Kiviat plots. This tab enables the user to change the default behavior of the tool, for example, to modify the ceiling and floor to insert color into the data, to adapt the placement of the buildings in the city map, etc. 

This tool may generate a report in the form of a LaTeX file, so the user can use it to produce a PDF file. This report summarizes all the values measured on the project. Each function and each file make a section of this report. In each section, numerical values are coupled with the signature of the function (as a Kiviat diagram). Each file produces three views of the associated city map diagram, as well as different Kiviat diagrams for the different metrics measured on the functions of the file. 

\section{Sextant}

Sextant is a Java source-code analysis tool under development at the University of Nebraska at Omaha (UNO)~\cite{sextant}. This software is a complex extension of the TL system (a general-purpose program transformation system) created particularly for the Java programming language domain.

The main design goal of Sextant is providing a tool facilitating specification and visualization of custom analysis rules, which can be domain-specific or moreover application specific analysis rules.

Analysis rules of Sextant are based on information fetched from different software models. There are two main models of central importance. The first model is a syntactic model. It is the source code parse tree. Parse trees conform to compilation units which represented by Java files and are generated with use of GLR parser technology provided by the TL system. Parse trees are well-fitted for analyzing and manipulating through standard primitives provided by program transformation systems, for example, by matching and generic traversal.

The second model is a compound attributed graph (CAG). It is a semantic model which captures subtype, structural, and reference dependencies among the constructors, methods, fields, packages and types. The CAG also links an attribute list with every node and edge.

CAG information is accessible to Sextant’s transformation-based analysis rules via two mechanisms. The first mechanism is a positional system which organizes a relation between contexts within the CAG and corresponding parse (sub)trees. This relation makes possible correctly resolving references to constructors, methods, types, fields, and local variables, during the process of generic traversals which are the key enabling mechanism in program transformation systems.

The second mechanism is a library of semantic queries. These queries can be accessed even in the middle of transformation course. Functionality which provided by this library contains things like:
\begin{itemize}
	\item Reference resolving.
	\item Reference type determining.
	\item Determining declaration shadowing or overriding.
	\item Determining whether one type is a subtype of another.
\end{itemize}
 
Sextant stores table and set types for information collecting with relation to analysis rules. These constructs can be used for storing information related to a custom metrics wide variety.

Sextant is open-ended with respect to the definition of metrics – any source-code analysis rule can be interpreted as a metric, be they PMD-style rules focusing on violations of coding conventions or rules such as those specified by FindBugs that are more semantic in nature\cite{sextant}.

Sextant can do software models generation. These models can be visualized using other tools such as GraphViz, Cytoscape and TreeMap. Sextant can produce the CAG of the code base in a JavaScript object notation format (JSON). This JSON file can be loaded into Cytoscape, an open source platform which provides extensive and sophisticated capabilities for large complex networks visualization, for example, graph structures. The same way, metrics derived from sets and tables can be produced in CSV format and viewed with use of TreeMap. Parse trees can be output as dot-files and later viewed with use of GraphViz.

The view in Figure\ref{fig:1} shows an example of represents a coloring of dependencies on the unsupported features. Nodes colored orange have indirect dependencies on unsupported features while purple nodes have direct dependencies on unsupported features.

\begin{figure}[h]
	\centering
	\includegraphics[height=70mm]{figures/1.png}
	\caption{An example of using Sextant tool.}
	\label{fig:1}
\end{figure}

\section{NDepend}

NDepend is a static analysis tool for .NET managed code. The tool supports a big number of code metrics, allowing to visualize dependencies using directed graphs and dependency matrix.  

NDepend computes a lot of size-related metrics: number of lines of code, number of assemblies, number of types, number of methods, etc. For measuring complexity, NDepend uses Cyclomatic Complexity. This metric measures the complexity of a type or a method by calculating the number of branching points in the code.
NDepend has two metrics for cohesion. Relational Cohesion is an assembly level metric that measures the average number of internal relationships per type. Lack of Cohesion Of Methods (LCOM) measures the cohesiveness of a type. A type is maximally cohesive if all methods use all instance fields.

NDepend uses a visualization tool called a Treemap.
NDepend comes with a dashboard to quickly visualize all application metrics s shown in Figure\ref{fig:dash}. The dashboard is available both in the Visual Studio extension. For each metric, the dashboard shows the diff since baseline. It also shows if the metric value gets better (in green) or wort (in red). 

\begin{figure}[h]
	\centering
	\includegraphics[height=70mm]{figures/dash.png}
	\caption{An example of using the dashboard.}
	\label{fig:dash}
\end{figure}

The Figure\ref{fig:tree} shows metric visualization using the colored treemap.

\begin{figure}[h]
	\centering
	\includegraphics[height=70mm]{figures/tree.png}
	\caption{An example of using the colored treemap.}
	\label{fig:tree}
\end{figure}

The tree structure used in NDepend treemap is the usual code hierarchy: 

\begin{itemize}
	\item .NET assemblies contain namespaces.
	\item Namespaces contain types.
	\item Types contain methods and fields.
\end{itemize}

\section{PVS-Studio}

PVS-Studio is a tool for detecting bugs and security weaknesses in the source code of programs, written in C, C++, and C\#. It works in Windows, Linux, and macOS environment~\cite{pvs}.

This tool executes static code analysis and after that creates a report for helping developers to find and fix bugs. PVS-Studio has wide range check methods. It helps to find misprints and Copy-Paste errors. The main value of static analysis is in its regular use so that errors are identified and fixed at the earliest stages~\cite{pvs}. 

PVS-Studio runs from the command line. The analysis results can be saved as HTML with full source code navigation. It is possible to not include files from the analysis by name, folder or mask; to run the analysis on the files modified during the last N days. Error statistics can be viewed in Excel~\cite{pvs}. 

This tool has an online reference guide concerning all the diagnostics available in the program, on the website and documentation. 

PVS-Studio has an integration with open source platform SonarQube designed for continuous analysis and measurement of code quality.

\chapter{Conclusion}
The main goal of this thesis, as stated at the beginning of this work, is to analyze a quality of programs written in the Erlang programing language by using RefactorErl tool. To accomplish that goal, it became necessary to provide a general overview of Erlang, RefactorErl static analysis tool and developed metrics. In this thesis the software quality, software metrics and some of tools for measure software quality metrics has been studied and analyzed.

Erlang is a functional language designed for highly parallel, scalable applications requiring high uptime. RefactorErl is a static source code analysis and transformation tool for Erlang providing several software metrics. The tool is developed by the Department of Programming Languages and Compilers at the Faculty of Informatics, Eötvös Loránd University, Budapest, Hungary. Among the features of RefactorErl is included a metric query language which can support Erlang developers in everyday tasks such as program comprehension, debugging, finding relationships among program parts, etc.

In this thesis we presented developed framework which analyze git repositories with Erlang code files. The new component is built on RefactorErl static analysis tool and actively uses its feature of calculating different metrics of Erlang modules and functions. This frawework allows to draw plots which show change of metrics with software evolving from version to version. The plots can be saved for future usage as a pictures in PNG format. 

The main focus of the component is to help Erlang developers with analyzing their projects using plots. Also, it was important to test the developed component on some projects and after that to analyze the measurments. For this purpose have been choosen three different projects from git. The experimental results allows to find changes (increasing the line of code number, char of code number, using otp library and etc.) in code. Visualisation helps with finding patterns and improving of the code quality.

It is safe to say that the main goals of this thesis were successfully achieved. The usage of software metrics is within an organization and it usage is expected to have a beneficial effect on software organizations by make software quality more visible.

%%%%%%%%%%%%%%%%%%%%%%%%%%%%%%
%% Bibliography starts here %%

\addcontentsline{toc}{chapter}{Bibliography}

%% There's more than one way to keep track of your citations.

%% For simply listing the citations by text you can use the thebibliography 
%% environment. See biblio.tex for an example. Comment out the following line
%% to use this style.
  
% %% You only need to keep one of biblio.bib and biblio.tex, depending which 
%% citation managament style you want to use.

\begin{thebibliography}{10}

\bibitem{ecore}
\newblock Az EMF Ecore meta-metamodell sematikus ábrája.
\newblock \emph{Eclipse Foundation. Xtext Documentation}
\newblock \\ https://eclipse.org/Xtext/documentation/308\_emf\_integration.html
\newblock \\ 2016.04.27

\bibitem{omguml}
\newblock \emph{Object Management Group. OMG Unified Modeling Language Superstructure}
\newblock \\ www.omg.org/spec/UML/
\newblock \\ 2015.11.30.

\bibitem{refactorerl1}
\newblock \emph{Melinda Tóth \& István Bozó:
\newblock Static Analysis of Complex Software Systems Implemented in Erlang},
\newblock \\ Central European Functional Programming Summer School Fourth Summer 
School, CEFP 2011, Revisited Selected Lectures, Lecture Notes in Computer
Science (LNCS), Vol. 7241, pages 451-514, Springer-Verlag, ISSN: 0302-9743,
\newblock 2012.

\bibitem{refactorerl2}
\newblock \emph{István Bozó, Dániel Horpácsi, Zoltán Horváth, Róbert Kitlei, Judit Kőszegi, Máté Tejfel, Melinda Tóth:
\newblock RefactorErl - Source Code Analysis and Refactoring in Erlang},
\newblock \\ Proceedings of the 12th Symposium on Programming Languages and Software
Tools, ISBN 978-9949-23-178-2, pages 138-148,
\newblock \\ Tallin, Estonia,
\newblock \\ 2011.10.

\bibitem{erlangdocs}
\emph{Erlang Programming Language},
\newblock \\ http://www.erlang.org/
\newblock \\ 2015.11.30

\end{thebibliography}



%% Another way is to use bibtex. The following command will process and 
%% include the citations listed in biblio.bib. The advantage of bibtex is that
%% you can simply copy-paste citations if the authors provided a bib-citation. 
%% For examples of such bib-citations, click the small "bib" link beside the 
%% articles at  https://plc.inf.elte.hu/erlang/refactorerl-academic-results.html

\bibliography{biblio.bib}{}
\bibliographystyle{unsrt}

%%%%%%%%%%%%%%%%%%%%%%%%%%%%
%% Appendices starts here %%

\addtocontents{toc}{\setcounter{tocdepth}{0}}
\end{document}
